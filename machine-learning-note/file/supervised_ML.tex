\section{教師あり機械学習}
全データを訓練データとテストデータの2種類に分ける.訓練データはトレーニングに用いるデータであり,テストデータは学習し終えたモデルをテストするためのデータである.

用意すべき訓練データとして,入力データ$\mathcal{X}=\{\vec{x}_1,\vec{x}_2,\ldots,\vec{x}_N\}$と出力データ$\mathcal{Y}=\{\vec{y}_1,\vec{y}_2,\ldots,\vec{y}_N\}$を用意する:$\mathcal{D}=\mathcal{X}\times\mathcal{Y}=\{(\vec{x}_1,\vec{y}_1),(\vec{x}_2,\vec{y}_2),\ldots,(\vec{x}_N,\vec{y}_N)\}$ここで,$\vec{x}_i\in\mathbb{R}^n$, $\vec{y}_i\in\mathbb{R}^m$である.機械学習の目的は,与えられた入力$\vec{x}$とその出力$\vec{y}$の関係を知ることである.つまり,
\begin{equation}
    \vec{y} = f(\vec{x})
\end{equation}
という関係において,$f$を知ることが目的である.ここでは,$N$個の訓練データがあるが,その中から1つだけデータを選び,それを$(\vec{x},\vec{y})$とおくこととする.
最も簡単な関数は,次である:
\begin{equation}
    \vec{y}=f(\vec{x}) = \hat{W}\vec{x} + \vec{b}
\end{equation}
\begin{equation}
    y_j= \sum_{i=1}^{n}w_{j,i}x_i + b_i
\end{equation}
ここで,$\vec{y}\in\mathbb{R}^m$, $\vec{x}\in\mathbb{R}^n$, $\hat{W}\in\mathbb{R}^{m\times n}$, $\vec{b}\in\mathbb{R}^m$である:
\begin{equation}
     \vec{y}=\left(
        \begin{array}{c}
       y_1\\[10pt]
       y_2\\[10pt]
       \vdots\\[10pt]
       y_m
        \end{array}
        \right),\ \ \ 
        \vec{x}=\left(
        \begin{array}{c}
          x_1\\[10pt]
       x_2\\[10pt]
       \vdots\\[10pt]
       x_n
        \end{array}
        \right),\ \ \ 
        \vec{b}=\left(
        \begin{array}{c}
          b_1\\[10pt]
       b_2\\[10pt]
       \vdots\\[10pt]
       b_m
        \end{array}
        \right)
\end{equation}

\begin{equation}
     \hat{W}=
     \left(
        \begin{array}{ccccc}
       w_{11}&w_{12}& \dots  & \dots& w_{1n}\\[5pt]
       w_{21}&w_{22}& \dots  & \dots& w_{2n}\\[5pt]
      \vdots&\vdots&\ddots &  & \vdots\\[5pt]
      \vdots&\vdots& &\ddots  & \vdots\\[5pt]
       w_{m1}&w_{m2}& \dots  & \dots& w_{mn}
        \end{array}
        \right)
\end{equation}

次に,次のような関数を考えてみよう.
\begin{equation}
    \vec{y} = \hat{W}_2\{\sigma(\hat{W}_1\vec{x}+\vec{b}_1)\} + \vec{b}_2
\end{equation}
ここで,$\vec{y}\in\mathbb{R}^m$, $\vec{x}\in\mathbb{R}^n$, $\hat{W}_1\in\mathbb{R}^{l\times n}$, $\hat{W}_2\in\mathbb{R}^{m\times l}$, $\vec{b}\in\mathbb{R}^m$, $\vec{b}_1\in\mathbb{R}^l$, $\vec{b}_2\in\mathbb{R}^m$である.また$\sigma(\cdot)$はベクトルに作用し,ベクトルを返す:
\begin{equation}
    \sigma(\vec{x}) = (\sigma(x_1)\ \sigma(x_2)\ \cdots\ \sigma(x_n))
\end{equation}
次に,また$\sigma(\cdot)$を重ねれば,
\begin{equation}
   \vec{y} = \sigma(\hat{W}_3\sigma(\hat{W}_2\sigma(\hat{W}_1\vec{x}+\vec{b}_1) + \vec{b}_2)+\vec{b}_3)
\end{equation}
となる.これは次のように漸化式で書くことができる:
\begin{equation}
    \vec{a}^{l} = \sigma(\hat{W}^{l}\vec{a}^{l-1} + \vec{b}^{l})
\end{equation}
と書ける.ここで,出力と入力(初期値)はそれぞれ$\vec{y}=\vec{a}^{L}$, $\vec{x}=\vec{a}^{0}$とおいた.さらに,非線形関数$\sigma(\cdot)$に通す前の出力を
\begin{equation}
    \vec{z}^{l} = \hat{W}^{l} \vec{a}^{l-1} + \vec{b}^{\ l-1}
\end{equation}
を定義すれば,
\begin{equation}
    \vec{y}=\vec{f}(\vec{x}) = \sigma(\vec{z}^{L}),\ \vec{a}^{0} = \vec{x}
\end{equation}
とまとめることができる.
これを$L$層のニューラルネットワークという.$l$層目の出力$\vec{a}^{l}$の要素$a_j^{l}$の添え字$j$はユニットを表している.

ニューラルネットワーク$f(\cdot)$はある入力$\vec{x}$があったとき,その出力が訓練データ$\vec{y}_i$にできるだけ近くなるように設計された方がよい.つまり,関数としては
\begin{equation}
    \vec{y} = f(\vec{x}) = \sigma(\vec{z}^{l}) \simeq \vec{y}_i
\end{equation}
となるような関数$f$が欲しい.このとき,次のをコスト関数,二乗誤差(Squared error)
\begin{equation}
    E = \|\vec{y}_i- f(\vec{x})\|
\end{equation}
が最小となるように$f$を構築すればよい.ここで,複数の訓練データについては,平均二乗誤差
\begin{equation}
    L(\vec{\theta}) = \frac{1}{N}\sum_{i = 1}^{N}\|\vec{y}_i - f_{\vec{\theta}}(\vec{x}_i)\| 
\end{equation}
をコスト関数として扱う.ここで,$\theta=(\hat{W},\ \vec{b})$であるから,$E$を最小化するための条件は
\begin{align}
    \frac{\partial E}{\partial w_{ij}^{l}}=0,\ \ \ \frac{\partial E}{\partial b_{j}^{l}}=0
\end{align}
であり,この条件を満たす,$\theta=(\hat{W},\ \vec{b})$を決めればよい.このようなパラメータを求めるためには,パラメータに関する微分が重要であることがわかる.





\subsection{}
\begin{equation}
    \hat{C}^{\rm{Ti}}=\Biggl(
    \bm{c}_1^{\rm{Ti}}\ \vec{c}_2^{\rm{Ti}}\ \bm{c}_1^{\rm{Ti}}\ \bm{c}_3^{\rm{Ti}}\ \bm{c}_4^{\rm{Ti}}\ \bm{c}_5^{\rm{Ti}}\ \bm{c}_6^{\rm{Ti}}\ \bm{c}_7^{\rm{Ti}}\ \bm{c}_8^{\rm{Ti}}
    \Biggr)
\end{equation}

\begin{equation}
    \hat{C}^{\rm{O}}
\end{equation}


