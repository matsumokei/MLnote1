\subsection{誤差逆伝播法}
上で述べたニューラルネットワークの最後の$L$層目の出力を具体的に見ると,
\begin{align}
    \vec{y} &= \sigma(\vec{z})^{L}\\[10pt]
    &=\sigma(
        \hat{W}^{(L)}
        \sigma(
            \hat{W}^{(L-1)}
            \sigma(
                \cdots
                \sigma(
                    \hat{W}^{(l+1)}
                    \sigma(
                        \textcolor{red}{\hat{W}^{(l)}}
                        \sigma(
                            \cdots
                        )
                    )
                )
            )
        )
    )
\end{align}
と多重合成関数となっていることがわかる.つまり,層が深くなればなるほど,巨大な合成関数を計算し,その微分を計算する必要がある.そこで,$E$のパラメータ微分を効率よく計算することが誤差逆伝搬法の目的である.

ニューラルネットワークにおいて$l$層目の$j$番目のユニットについて考える.まず,次のデルタを導入する:
\begin{equation}
    \delta_{j}^{l} \equiv 
    \frac{\partial E}{\partial z^{l}_{j}}
\end{equation}
これは,$l$層目における$j$番目の誤差を表す.$\vec{z}^{(l)}$の定義は
\begin{equation}
    \vec{z}^{(l)}
    =\hat{W}^{(l)} \vec{a}^{(l-1)} + \vec{b}^{(l)}
\end{equation}
\begin{equation}
    z^{(l)}_{j}
    =\sum_{k}w^{(l)}_{j,k} a^{(l-1)}_{k} + b^{(l)}_{j}
\end{equation}
であったから,パラメータ$\hat{W}$,$\vec{b}$に関する微分はそれぞれ,
\begin{align}
    \frac{\partial E}{\partial b^{l}_{j}}
    &=\sum_{k}\frac{\partial z^{l}_k}{\partial b^{l}_{j}}
    \frac{\partial E}{\partial z^{l}_{k}}\nn[10pt]
    &=\sum_{k}\delta_{k,j}
    \frac{\partial E}{\partial z^{l}_{k}}
    =\frac{\partial E}{\partial z^{l}_{j}}=\delta^{l}_j,
\end{align}

\begin{align}
    \frac{\partial E}{\partial w^{l}_{i,j}}
    &=\sum_{k}\frac{\partial z^{l}_k}{\partial w^{l}_{i,j}}
    \frac{\partial E}{\partial z^{l}_{k}}\nn[10pt]
    &=\sum_{k}\frac{\partial E}{\partial z^{l}_{k}}
    \delta_{k,i} a^{l-1}_{j}
    =\frac{\partial E}{\partial z^{l}_{i}} a^{l-1}_{j}
    =\delta^{l}_{i}a^{l-1}_{j}
\end{align}
となる.ここで,
\begin{equation}
    z^{l}_{k} = \sum_{s}w^{(l)}_{k,s} a^{(l-1)}_{s} + b^{(l)}_{k}
\end{equation}
であるので,
\begin{equation}
    \frac{\partial z^{l}_k}{\partial w^{l}_{i,j}}
    = \sum_{s}\frac{\partial w^{l}_{k,s}}{\partial w^{l}_{i,j}}
    a^{(l-1)}_{s}
    =\frac{\partial w^{l}_{k,j}}{\partial w^{l}_{i,j}}
    a^{(l-1)}_{j}
    =\delta_{k,i}a^{l-1}_{j}
\end{equation}
が成り立つからこれを用いた.つまり,デルタ$\delta^{l}_{j}$を計算することができれば,パラメータ微分を実行できる.ここで,$a^{l-1}_{j}$は順伝播ですでに計算し保存している値を使えば良い.


まず,$L$層目のデルタ$\delta^{L}_j$について計算を行う.
\begin{align}
    \delta^{L}_{j}
    =\frac{\partial E}{\partial z^{L}_{j}}
    =\sum_{k}\frac{\partial a^{L}_{k}}{\partial z^{L}_{j}}
    \frac{\partial E}{\partial a^{L}_{k}}
\end{align}
ここで,$\vec{a}^{L}=\sigma(\vec{z}^{L})$であるから,
\begin{align}
    \delta^{L}_{j}
    &=\sum_{k}\delta_{k,j}\frac{\partial a^{L}_{k}}{\partial z^{L}_{j}}
    \frac{\partial E}{\partial a^{L}_{k}}\nn[10pt]
    &=\frac{\partial a^{L}_{j}}{\partial z^{L}_{j}}
    \frac{\partial E}{\partial a^{L}_{k}}
    =\frac{\partial \sigma(z^{L}_{j})}{\partial z^{L}_{j}}
    \frac{\partial E}{\partial a^{L}_{k}}
    =\frac{\partial E}{\partial a^{L}_{k}}\sigma^{\prime}(z_j^{L})
\end{align}
と計算できる.関数$\sigma(\cdot)$の微分は解析的に実行可能である.次に,$l$層目でのデルタ$\delta^{l}_{j}$を求める.
\begin{align}
    \delta^{l}_{j}
    =\frac{\partial E}{\partial z^{l}_{j}}
    =\sum_{k}\frac{\partial z^{l+1}_{k}}{\partial z^{l}_{j}}
    \frac{\partial E}{\partial z^{l+1}_{k}}
    =\sum_{k}\frac{\partial z^{l+1}_{k}}{\partial z^{l}_{j}}
    \delta^{l+1}_k
\end{align}
となる.さらに,
\begin{equation}
    \vec{z}^{(l)}
    =\hat{W}^{(l)} \vec{a}^{(l-1)} + \vec{b}^{(l)}
    =\hat{W}^{(l)} \sigma(\vec{z}^{\ l-1}) + \vec{b}^{(l)}
\end{equation}
を使えば,
\begin{equation}
    \frac{\partial z^{l+1}_{k}}{\partial z^{l}_{j}}
    =w^{l+1}_{k,j}\sigma^{\prime}(z_j^{l})
\end{equation}
となり,
\begin{align}
    \delta^{l}_{j}
    &=\sum_{k}w^{l+1}_{k,j}\sigma^{\prime}(z_j^{l})
    \delta^{l+1}_k\nn[10pt]
    &=\sigma^{\prime}(z_j^{l})\sum_{k}w^{l+1}_{k,j}
    \delta^{l+1}_k
\end{align}
を得る.この式から,$l+1$層目のデルタを計算すれば,$l$層目のデルタを計算できることがわかる.ここで,Eq.~\eqref{}の計算は次のように計算できる:
\begin{equation}
    z^{l+1}_{k} = \sum_{s}w^{(l+1)}_{k,s} a^{(l)}_{s} + b^{(l+1)}_{k}
\end{equation}
であるので,
\begin{equation}
    \frac{\partial z^{l+1}_k}{\partial z^{l}_{j}}
    = \sum_{s}w^{(l+1)}_{k,s} 
    \frac{\partial a^{(l)}_{s}}{\partial z^{l}_{j}}
    = \sum_{s}w^{(l+1)}_{k,s} 
    \frac{\partial \sigma(z^{(l)}_{s})}{\partial z^{l}_{j}}
    =w^{l+1}_{k,j}\sigma^{\prime}(z^{l}_j)
\end{equation}
